\noindent In summary, the findings of this study have gathered a few strong conclusions from this research project which could act as a stepping stone to formulate a simplified yet robust reduced-order model for future investigations. Firstly, hydrodynamic diameter is the preferable input parameter for apparent viscosity estimations compared to geometric diameter. Secondly, RBC distribution within studied networks is flow-mediated and the presence of plasma skimming effect was demonstrated from simulation data. Thirdly, the distance ratio ($L$/D$_{H}$) could be an additional contributing factor to relative apparent viscosity which explains the development of CFL recovery. Hence, this will influence the asymmetric RBC distribution across the network apart from the known effects of D$_{H}$ and H$_{D}$ on apparent viscosity. Last but not least, the heterogeneity of RBC distribution is attributed to the upstream perturbation in the CFL and the lack of CFL recovery between consecutive bifurcations. Therefore, with the accumulation of high-fidelity simulation data of blood flow in complex microvascular networks, a robust reduced-order model or extended constitutive models can be developed to improve the estimation of apparent viscosities. This can provide more reliable flow resistance estimations within a network while minimising the requirement for computational power. Furthermore, we will be able to gain a better understanding of the RBC distribution mechanisms at the micro-circulation level and characterise the cellular character embedded in the general blood flow. \\

\noindent Nevertheless, there are still numerous missing gaps that should be looked into for future works in order to achieve a robust reduced-order model that will not only aid future investigations but also clarify how certain effects propagate at an entire network level. Some of the considerations are:

\begin{enumerate}
    \item Consider the presence of WBCs or platelets in blood flow or the presence of endothelial surface layer on vessel walls in the networks.
    \item Consideration of other plausible mechanisms such as separation surface cross-over, lingering and jamming effects of RBCs at the bifurcation points. 
    \item Increase haematocrit in simulation to validate the importance of hydrodynamic diameter over geometric diameter in Pries's PSM and viscosity models. 
    \item Suggestions on how to quantify the blood flow entrance effect with $L$/D$_{H}$:
    \begin{itemize}
        \item Quantification of CFL thickness will be required to correlate with $L$/D$_{H}$ in order to justify the relative significance of the CFL in each vessel branch .
        \item Identify the critical threshold of $L$/D$_{H}$ with haematocrit for apparent viscosity between simulation data and empirical predictions across the networks.  
    \end{itemize}
\end{enumerate}



% \section{Emerging Challenges \& Opportunities}
% \begin{itemize}
%     \item Find out number of occurrences of the lingering/jamming effects in the 3D complex blood flow networks. 
%     \item 
% \end{itemize}




% discover new insights and phenomena in microcirculatory blood flow

% Last but not least, 
% This allows us to formulate reduced-order models that are capable of predicting and describing the haemodynamics in microcirculations. 
