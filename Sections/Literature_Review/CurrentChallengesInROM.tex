\noindent Over the years as the development of computational techniques have improved significantly, a plethora of computational studies\cite{Noguchi14159, DoddiSaiK2009Tcmo, Freund2014, Balogh2017DirectNetworks, 2020Charles, YAN199117, Possenti2019, EndenG1994ANSo, MehrabadiMarmar2015ACMf, VahidkhahKoohyar2014PDiT, HariprasadSecomb2014, Zhou2021EmergentBifurcations} have been conducted to elucidate and analyse the haemodynamics of blood flow in microvascular networks. To overcome the limitations of 2D systems, numerous 3D models emerged over recent years (E.g. immersed-boundary-lattice-Boltzmann model) to provide a reliable quantitative description of the microscopic behaviour of RBCs in micro- circulations.\cite{KrugerTimm2012Csso} However, conducting these studies required a lot of computational expense (i.e. GPU power) which hinders researchers to simulate RBC flows in large computational domains due to computational tractability limitations. Therefore, a great number of reduced-order models (ROM) were introduced to simplify the mathematical calculations in a computationally efficient manner and to overcome this limitation.\cite{A.R.Pries2005Mbvi, Romain2020, Bernabeu2020AbnormalOxygenation, Gould2015HematocritNetworks, PRIES198981, PriesAR1994RtBF} The main objective for ROMs is to quantitatively analyse and predict haemodynamic quantities with satisfactory accuracy in complex microvascular networks. \\

\noindent The validity of these ROMs is often restricted to a limited range due to a few conventional assumptions that are only applicable to certain conditions such as Poiseuille's Law. One example is from Pries's Phase Separation Model (PSM)\cite{A.R.Pries2005Mbvi, PriesAR1990BFiM, PRIES198981} which has the assumption of symmetrical haematocrit distribution in the feeding vessel. This assumption is usually invalid for actual micro-vessels given that the RBC distribution in these vessels was often observed to be asymmetric.\cite{2020Charles, Balogh2017DirectNetworks, Barber2011SimulatedPartitioning, Roman2016} Furthermore, most of these well-established ROMs are derived from experiments under certain assumptions and do not entirely quantify the true representation of the physiological phenomena associated with the motion of RBCs in complex microvascular networks. One example would be Pries's Viscosity Models (\textit{in vivo}\cite{PriesAR1994RtBF} and \textit{in vitro}\cite{Pries1992BloodHematocrit}) which are derived from experimental. This limits the capability of predicting the blood flow behaviour or mechanism that governs the radial distribution of RBCs in micro-vessels across the network at different scales. Therefore, further progress is required to better understand the fundamentals of the physical mechanisms in RBC flux partitioning. This will allow us to achieve reliable predictions of the physiological blood flow behaviours in both micro-circulation without simulating the motion of every individual RBC in large-scale numerical simulations. 