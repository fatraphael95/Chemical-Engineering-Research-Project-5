\noindent Following up with recent studies, only simulations of 2D simplified geometries or idealised 3D microvascular networks instead of realistic blood topology were used to analyse the physical mechanisms occurring in micro-circulations.\cite{Noguchi14159, DoddiSaiK2009Tcmo, Balogh2017DirectNetworks, 2020Charles, YAN199117, Possenti2019, EndenG1994ANSo, MehrabadiMarmar2015ACMf, VahidkhahKoohyar2014PDiT, HariprasadSecomb2014, Zhou2021EmergentBifurcations} This restricts the full potential of identifying or demonstrating certain fundamental effects in complex microvascular networks due to certain limitations. One example would be simulating 2D systems with only 1 particle as it will neglect the effects of cell-to-cell interactions and these 2D models are not able to accurately replicate the data from realistic 3D systems.\cite{PhysRevE.86.056308} Furthermore, for 2D systems, it is often assumed that there is no overlap in cell position along with uniform and independently distributed arrival times.\cite{Balogh2017DirectNetworks} This is not true for 3D complex network geometries as the RBCs travelling along the micro-vessels tend to form clusters due to the heterogeneity in their shape, size and deformability.\cite{gaehtgens1980motion} \\

\noindent The majority of the simulations consider the RBC properties of a normal healthy cell and no other suspended particles such as white blood cells (WBC) or platelets were considered in blood.\cite{Balogh2017DirectNetworks, 2020Charles, Balogh2018} The presence of WBCs or platelets are very likely to affect the RBC partitioning behaviours across the microvascular networks due to their important biological functions and cellular interactions. Also, based on the physical perspectives, considerations of including more biophysical properties of the RBCs (E.g. viscosity of RBCs and its membrane) will bridge us closer to simulate the actual blood flow behaviours in micro-circulation under physiological conditions. This also applies to the deformation and regulation of blood vessels in the simulated 3D complex networks. Therefore, this would allow us to attain more reliable results from the investigations of micro-circulatory blood flow and a deeper understanding of the hydrodynamic interactions of RBCs in blood flow. Last but not least, it will also significantly advance the wider biomedical and biophysical applications such as optimising micro-fluidic devices. 