\noindent Numerous \textit{in vivo} and \textit{in vitro} studies\cite{LIPOWSKY1980297, PRIES198981, CARR1990179, PriesAR1990BFiM, PriesAR1994RtBF, SmithMichaelL2003Nmra, Sherwood2014, Roman2016, Mantegazza2020} have been conducted over many decades to examine blood flow behaviours and investigate the partitioning of RBCs at vascular bifurcations. Many of these studies have demonstrated that across the networks, the RBCs are often not distributed with the same proportion as the blood flow. However, both methods of scientific studies are very different because \textit{in vivo} studies refers to performing experiments within a living organism (i.e. real blood vessel segments) whereas, for \textit{in vitro} studies, it refers to conducting experiments in a controlled environment such as glass capillary tubes. This affects the interpretation of both results given that the \textit{in vitro} experiments cannot replicate the conditions that occur inside a living organism. Therefore, it is frequently observed that the velocity profiles, haematocrit distributions and blood viscosity do vary substantially between \textit{in vivo} and \textit{in vitro} experiments. \\

\noindent For \textit{in vitro} studies, most of these experiments using long straight glass tubes generally produce symmetrical RBC distributions, unlike \textit{in vivo} studies where asymmetric RBC distributions are more frequently observed.\cite{Sherwood2014} In addition to this, blood viscosity was often found to be several factors lower in \textit{vitro} compared to \textit{in vivo} environments.\cite{Pries2000TheLayer, A.R.Pries2005Mbvi} This is due to the presence of endothelial surface layer (ESL) found within the interior blood vessel walls, unlike \textit{in vitro} experiments where the glass capillary tubes have smooth interior surfaces. The ESL consists of squamous endothelial cells, which reduces the effective width of the lumen and largely hinders the flow of plasma and RBCs.\cite{Pries2000TheLayer, WeinbaumSheldon2007Tsaf} Therefore, the results obtained from both \textit{in vivo} and \textit{in vitro} experiments have indicated that the significant difference in the measured flow resistance values is caused by ESL. This makes it very challenging to accurately model the blood flow behaviours and RBC distributions in micro-circulations. \\

\noindent To summarise, \textit{in vivo} experiments are done within living organism under physiological conditions which makes it time-consuming and expensive to perform. In comparison, \textit{in vitro} experiments refers to the technique of performing a given procedure in controlled laboratory conditions that requires lesser time and is cheaper to perform but are limited to their functions as they will need further validation from \textit{in vivo} results. 