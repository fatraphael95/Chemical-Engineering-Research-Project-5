\begin{abstract}

%% Context
%% Research Focus
%% Methodology 
%% Main Results 
%% Conclusions

\noindent Diseases such as cancer and cardiovascular disease frequently and adversely affect the global population and these conditions are commonly associated with blood and micro-circulatory disorders. Despite current treatments, there is a need for more advanced medical treatments and therapies for disease prevention and patient treatment. Unfortunately, there have been significant challenges in quantitatively describing the blood flow behaviours observed in the micro-vessels due to the complexity of micro-circulatory blood flow in the human body. Furthermore, there may be gaps in the understanding of haemodynamic mechanisms in micro-scale systems where computational modelling could be a potent tool to elucidate the observed blood flow behaviours in micro-circulation. In this study, the existing particle-based simulation data of blood flow in 3D complex microvascular networks were investigated. The evaluated predictions from several established reduced-order models were then analysed against simulation data to identify the contributing factors for the observable effects. The results from this study demonstrate the existence of plasma skimming effects in the microvascular networks, therefore implying that the RBC distribution within the studied networks is flow-mediated. The results have also shown that the hydrodynamic diameter is a preferable input parameter for apparent viscosity estimations. Furthermore, it was observed that the distance ratio ($L$/D$_H$) could be an additional contributing factor to relative apparent viscosity as this influences the asymmetric RBC distribution across the network. Based on these findings, these insights can act as building blocks to formulate a simplified yet robust reduced-order model for future investigations. In summary, this contributes to attaining a better understanding of the radial distribution of RBCs in micro-vessels based on fluid mechanical perspectives. 




% \noindent In this day and age, certain medical conditions such as cancer and various cardiovascular diseases have been occurring more frequently around the world and these diseases are all associated with blood and micro-circulatory disorders. This signals the need for more advanced medical treatments and therapies to heal the patients or delay the progression of the disease. However, there are still significant challenges in quantitatively describing the blood flow behaviours observed in the micro-vessels due to the complexity of micro-circulatory blood flow in the human body. There might also be other missing details of the haemodynamic mechanisms in micro-scale systems that have not been discovered to explain the observed blood flow behaviours in micro-circulation. In this study, existing particle-based simulation data of blood flow in 3D complex microvascular networks were investigated and the evaluated predictions from several established reduced-order models were analysed against simulation data to identify the contributing factors for the observable effects. The results from this study have demonstrated the existence of plasma skimming effects in the microvascular networks and the RBC distribution within the studied networks is flow-mediated. Furthermore, it was proven that the hydrodynamic diameter is the preferable input parameter for apparent viscosity estimations and the distance ratio ($L$/D$_H$) could be an additional contributing factor to relative apparent viscosity as this influences the asymmetric RBC distribution across the network. Based on these findings, it can act as a stepping stone to formulate a simplified yet robust reduced-order model for future investigations to attain a better understanding of the radial distribution of RBCs in micro-vessels based on fluid mechanical perspectives. 


\end{abstract}