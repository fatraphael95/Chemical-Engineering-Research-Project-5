\noindent The Phase Separation Model (PSM) consists of a list of empirical equations that were derived from experimental observations of arteriolar bifurcations in a rat mesentery and was developed by Pries et al.\cite{A.R.Pries2005Mbvi, PriesAR1990BFiM} through theoretical modelling. The PSM has established a flow-mediated mechanism that could quantitatively predict the amount of RBC fluxes in the child branches of diverging bifurcations within a network of micro-vessels. The fractional RBC flux (\textit{FQ$_{E}$}) in each child branch is given as a function of the fractional blood flow (\textit{FQ$_{B}$}) entering that child branch:

\begin{eqnarray}
\label{Pries_equation1}
\begin{aligned}
logit FQ_{E} & = A + B \thinspace logit \frac{(FQ_{B} - X_{0})}{(1 - 2X_{0})}
\end{aligned}
\end{eqnarray}

\bigskip

\noindent where $logit$ $x$ $=$ ln($\frac{x}{1-x}$) and X$_{0}$ < FQ$_{B}$ < 1 $-$ X$_{0}$. The fitting parameters $A$, $B$ and $X_{0}$ were obtained via linear regression analysis. $A$ represents the size difference of both child branches, $B$ describes the shape of the haematocrit profile in the parent branch and $X_{0}$ is associated with the cell-free layer thickness near the inner branch walls. \\

\begin{eqnarray}
\label{Pries_equationA}
\begin{aligned}
A & = -13.29\bigg[\bigg(\frac{D_{\alpha}^{2}}{D_{\beta}^{2} -1}\bigg) \bigg/ \bigg(\frac{D_{\alpha}^{2}}{D_{\beta}^{2} +1}\bigg)\bigg]\bigg[\frac{(1 - H_{D,PB})}{D_{PB}}\bigg]
\end{aligned}
\end{eqnarray}

\begin{eqnarray}
\label{Pries_equationB}
\begin{aligned}
B & = 1 + 6.98\bigg[\frac{1 - H_{D,PB}}{D_{PB}}\bigg]
\end{aligned}
\end{eqnarray}

\begin{eqnarray}
\label{Pries_equationX}
\begin{aligned}
X_{0} & = 0.964\bigg[\frac{1-H_{D,PB}}{D_{PB}}\bigg]
\end{aligned}
\end{eqnarray}

\bigskip

\noindent where $D_{\alpha}$, $D_{\beta}$, and $D_{PB}$ are the diameters of two child branches and the parent branch respectively while $H_{D,PB}$ represents the discharge haematocrit in the parent branch. The simulation data in the form of $Q^{*}_{rbc}$ against $Q^{*}_{blood}$ are plotted with the empirical curves for each bifurcation in order to observe the deviations between simulation data and the empirical predictions from PSM. \\

\noindent The following points are some key notes for Pries's Phase Separation Model:\cite{A.R.Pries2005Mbvi, PriesAR1990BFiM, PRIES198981}
\begin{itemize}
    \item Assumes haematocrit profile in the parent branch of a diverging bifurcation is symmetrical
    % \item Planar flow separation surface between the fluid spaces entering the child branches (A = 0)
    \item Velocity profile in the parent branch was assumed to be parabolic
    \item Under dilute/semi-dilute suspensions, the RBCs are assumed to follow the fluid streamlines (i.e. plasma) in the vicinity of the bifurcation
    \item Empirical model does not take into account the bifurcation angle as it was proven to be of minor importance. 
\end{itemize}

% Put the key notes in a paragraph instead of bullet points % 